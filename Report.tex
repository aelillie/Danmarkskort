\documentclass[a4paper]{article}
\usepackage{hyperref}
\usepackage[utf8]{inputenc}
\usepackage[danish]{babel}
\usepackage[T1]{fontenc}
\usepackage{graphicx}
\usepackage{times}
\usepackage{booktabs}

%Preample
\begin{document}

\section{Forord og indledning}
Denne rapport er udarbejdet af gruppe G i foråret 2015 i forbindelse med kurset Førsteårsprojekt: Danmarkskort på IT-Universitetet i København under vejledning af Troels Bjerre Sørensen. Projektet omhandler programmering af et interaktivt danmarkskort, som man kan benytte til at søge og navigere til og fra adresser. Programmet behandler data hentet fra Open Street Maps XML-filer, som visualiseres gennem brugergrænsefladen. 

\section{Baggrund og problemstilling}
Under udarbejdelse af et softwaresystem, der behandler samt visualiserer geografisk data, er der flere situationer, der skal overvejes. Der kan udtænkes flere scenarier, som en bruger af programmet vil kunne efterspørge med henblik på søgning, navigation, ruteplanlægning og visualisering. 

\subsection{Navigation og ruteplanlægning}
En af de hyppigste og vigtigste opgaver, som det ovenfor beskrevne  system skal kunne løse, er blandt andet at tillade søgning på lokationer og specificere en valgt lokations placering  visuelt i programmet. Herunder er det et hyppigt scenarie, at brugeren ønsker rutevejledning fra vedkommendes nuværende placering til en valgt adresse, og programmet skal derfor kunne præsentere en rutevejledning fra A til B. Hvis brugeren ønsker rutevejledning, skal det også være muligt at vælge hvilken slags transportform, der skal benyttes, f.eks. bil, cykel eller gåben. 
Programmet skal foreslå den hurtigste rute afhængigt af den valgte transportform, eksempelvis er det ofte hurtigst at køre på motorvej, hvis man er i bil, men når man er på cykel eller gang, skal motorvejene ikke inkluderes i ruten.

\subsection{Visualisering}


\section{Problemanalyse}


\end{document}
